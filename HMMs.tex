The following is a brief description of hidden Markov Models (HMMs) adapted from \cite{mesa2016hidden}.

Let $x = (x_1,x_2,\ldots,x_n)$ be a sequence of observations where each of the residues $x_j$ takes values from a finite alphabet $D$. 
Also $y = (y_1,y_2,\ldots,y_n)$ is a collection of states, where each $y_i$ belongs to $\mathcal{E}$, a finite set of states. The sequence $x$ can be thought of as a time series where the indeces $j$ are taken as discrete time steps. These models are considered 'hidden' because only sequence of residues $x$ is observable and but the sequence of states is unkown.

The model starts at some initial state, produces a residue and moves to a new state and procudes a new residue. This continues until the end of the sequence. Let $P(x_i|y_i = k)$ denote the probability that the emission $x_i$ was produced by state $k$ at time $i$. Next, let the probability that a transition is made from state $k$ to state $l$ given by $P(y_{i+1} = l | y_i = k)$. The joint distribution for the probability the model transits path $y$ and generates the sequence $x$ is then
	\begin{equation}
		P(x,y) = \prod_{i=1}^{n} P(y_i|y_{i-1})P(x_i|y_i),
	\end{equation}
where $P(y_i|y_{i-1}) = P(y_{i} = l | y_{i-1} = k)$ and $P(x_i|y_i) = P(x_i|y_i = k)$. 

The parameters of the model are the transisition probabilities and the emission probabilities. In practice, these parameters are estimated, for example with Expectation Maximization.

There are three assumptions underlying the model and they are:
	\begin{itemize}
		\item The Markov property holds for the states $y_i$. That is $P(y_{i+1}|y_i,\ldots,y_1) = P(y_i|y_{i-1})$.
		\item The Markov chain produced by the states $y$ is homogeneous. That is the probability of transitioning from one state to another is independent of time $i$. 
		\item Conditional independence of observations. That is $x_i$ is only dependant on the state $y_i$. 
	\end{itemize}

An example of usage would be to assign the alphabet $D = \{A,C,G,T\}$ representing the four nucleotide bases found in DNA and the states as $\mathcal{E} = \{\text{intron}, \text{exon}, \text{intergenic region}\}$. 
% These models have been used in computational biology \textbf{cite?}

%albus, silas, salus, meris, janus, miles, myles, regis, remus  





