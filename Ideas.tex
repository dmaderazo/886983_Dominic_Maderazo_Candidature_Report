Use the methylation data to infer the chromatin states of each organism. Methylation data has been used to infer models of transcriptional regulation~\cite{duren2017modeling}. Combine the DNA sequence here with the methylation data. 
% What sort of prior would one place on this? How does one choose to map some biological explanation with some observed phenomenon. Well, we are able to say that these particular positions are methylated. 
Perhaps the methylation marks can be used to produce a profile of the chromatin state. 
Among the sequences located in regions that are inferred to be in open chromatin, we form a library for each organism, identify motifs that are candidates for Enhancers. It may be the case that general TFBSs are identified and further work needs to be done to further classify something as an enhancer. Alignment free sequence comparison can be maybe used to see what motifs are conserved. 

The methylation data can act as a filter for sequences in the open state and this is important because this profile will vary according to the cell type and function and to a degree, species. We have a focus on cardiac cells at various stages of development, since this is the data vailable to us and of interest to our collaborators. These differing profiles can inform us on the enhancer regions that are conserved across species. 

See if there exists some pre-existing work that models chromatin state based on methylation data in a bayesian approach. Then you can judge model assumptions, select the ones that you agree with and reasses the ones that you are less comfortable with. 

Hopefully at the bare minimum something like that exists. If it doesn't, see what anyone has done with ChIP-Seq data in a bayesian setting. Maybe even start comparing some frequentist approaches and see what comparisons with bayesian inferrences and how this may translate over to our context. If no ChIP-Seq methods have been used, compare with previous generation sequencing think about how these inferences are robust when considering the difference getting to comparable ChIP-Seq stuff. 

Notes to self:

Where is the heirarchy? How deep will it have to go? What will you do to change assumptionstha are not relevant to you? That will depend on the prevalence of Bayesian methodologies in the field of bioinfomatics and next generation sequencing. I suspect some amount of work will have been done at this point in time. It is daunting to think there is nothing of the sort and I am the one to pioneer some sort of approach to this problem. There is some allure to it, but I am not confident it wil be such. 