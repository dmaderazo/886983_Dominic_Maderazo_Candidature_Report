\subsection{Alignment Based Enhancer Identification}

Change Point has been used in the past to identify non-coding putative functional elements (ncPFEs) that have been conserved across species. We seek a way to extend the method to incorporate data made available to us and further classify these elements according to their function. Specifically, we desire to be able to identify enhancer elements and, if possible, distinguish those that are tissue specific.

Using an unpublished manuscript started by a previous PhD candidate and their data, I will attempt to recreate the results they obtained; a set of ncPFEs. Once these ncPFEs are obtained, I will then attempt to classify the elements by using a combination of binary classifiers about features associated enhancer regions in an application of the idea presented by Keith and Boyd~\cite{keith2012bayesian}. I intend to incorporate the methylation data provided by our collaborators into the binary classification. The results of this analysis will then be compared against existing databases of known elements to evaluate the efficacy of this approach. 

One of the longer term goals of this project is to publish the scripts needed for the model selection, analysis and visualization for Change Point and binary classification in the computing language R as a package with an accompanying publication.

% Then, I plan on applying using our collaborator's methylation data in the form of a binary classifiers combined withite{keith2012bayesian}

\subsection{Alignment Free Enhancer Identification}

In the literature reviewed there does not appear to be an alignment free method that for the identification of enhancers that incorporates epigenetic markers, such as methylation data. For the second project, I would like to investigate the viability of this method on the genome scale. My proposed modification to the alignment free would be a weighting in the calculation for comparison statistic used e.g. weighting the terms on the RHS of Equation~\ref{eqn:AF} depending on whether the region they are located in is methylated or not. This may have the affect of biasing the the sequence comparison methods to regions associated with certain epigenetic markers, such as those where active or primed enhancers may be found.

At this stage, I am still developing my understanding of alignment free comparison methods that are available. I will attempt to identify cardiac enhancers using alignment free methods, incorporating methylation data.
If an extension to an existing method is developed or a completely new one is created, the aim is also to publish this methodology of process in a journal paper and any software to be available open source.