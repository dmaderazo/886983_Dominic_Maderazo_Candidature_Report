\documentclass[12pt,a4paper]{article}
\author{D Maderazo}
\usepackage{amsmath, amsthm, amssymb, textcomp, enumerate, multicol, fancyhdr, varwidth, graphicx, color, mathrsfs, tikz, pgfgantt}
\usepackage[left=25mm,right=25mm,top=30mm,bottom=30mm]{geometry}
\usepackage{amsfonts, float}
\usepackage[]{algorithm2e}
\usepackage{csquotes}

\usepackage{xcolor,colortbl}

\linespread{1}
\newtheorem*{mydef}{Definition}
\newtheorem{mythm}{Theorem}
\pagestyle{fancy}
\lhead{}
\rhead{Dominic Maderazo}
\newcommand*\dee{\mathop{}\!\mathrm{d}}
\newcommand*\del{\mathop{}\!\partial}
\def\*#1{\mathbf{#1}}
\setlength{\parskip}{1em}
\usetikzlibrary{shapes.geometric, arrows}
\tikzstyle{startstop} = [rectangle, rounded corners, minimum width=3cm, minimum height=1cm,text centered, draw=black, fill=red!30]
\tikzstyle{io} = [trapezium, trapezium left angle=70, trapezium right angle=110, minimum width=3cm, minimum height=1cm, text centered, draw=black, fill=blue!30]
\tikzstyle{process} = [rectangle, minimum width=3cm, minimum height=1cm, text centered, draw=black, fill=orange!30]
\tikzstyle{decision} = [diamond, minimum width=3cm, minimum height=1cm, text centered, draw=black, fill=green!30]
\tikzstyle{arrow} = [thick,->,>=stealth]

\usepackage{forloop}
\newcounter{loopcntr}
\newcommand{\rpt}[2][1]{%
  \forloop{loopcntr}{0}{\value{loopcntr}<#1}{#2}%
}
\newcommand{\on}[1][1]{
  \forloop{loopcntr}{0}{\value{loopcntr}<#1}{&\cellcolor{gray}}
}
\newcommand{\off}[1][1]{
  \forloop{loopcntr}{0}{\value{loopcntr}<#1}{&}
}


% \linespread{2}
% \setlength{\parindent}{4em}        
\begin{document}
% \title{Application of Bayesian Techniques for Genome Segmentation Analysis}
% \date{\today}
% \author{Dominic Maderazo}
% \begin{center}
%     {Supervisors: Jennifer Flegg \& Jonathan Keith}
% \end{center}%Supervisors: Jennifer Flegg \& Jonathan Keith}
% \renewcommand{\maketitlehookb}{\centering You won't expect the results}
% \maketitle
\begin{titlepage}
	\centering
	\includegraphics[width=0.15\textwidth]{example-image-1x1}\par\vspace{1cm}
	{\scshape\LARGE Columbidae University \par}
	\vspace{1cm}
	{\scshape\Large Final year project\par}
	\vspace{1.5cm}
	{\huge\bfseries Pigeons love doves\par}
	\vspace{2cm}
	{\Large\itshape John Birdwatch\par}
	\vfill
	supervised by\par
	Dr.~Mark \textsc{Brown}

	\vfill

% Bottom of the page
	{\large \today\par}
\end{titlepage}
    % \section*{Abstract}
    % Yes
    % sub Sections for lit review: (levels *, and > )
    %     * Bio background
    %     * Enhancer Identification
    %     * Mathematics
    %         > Param Est
    %         > Markov Chains
    %         > HMMs
    %         > Alignment
    %         > Change Point
    %         > Heirarchical modelling?
    %         > MEME
    

    \section{Literature Review}
        \subsection{Biological Background}
                \subsection{Biological Background}
    % Transcription factors are important because they modify the transcriptional machinery
        Gene expression is crucial to the function and differentiation of different cell types and required for growth, repair and maintenance of an organism.  In eukaryotic cells, DNA is coiled around a protein structure called a histone; these are collected in tightly packed configurations of DNA, known as chromatin, in the nucleus of cells~\cite{alberts2002chromosomal, kornberg1974chromatin}. A gene is a region in the DNA that contains the necessary information for the synthesis of an associated protein. The types of genes and the intensity associated with their expression in a cell is known as an expression profile. All the nucleated cells in an organism contain the same DNA, but the differences in the expression profiles is what gives rise to the different cell types and functions~\cite{lockhart2000genomics}. Our interest lies in the identification and classification of certain regions in the DNA that help regulate the expression of particular family of cardiac genes. 
        
        One of the enduring tenets of molecular biology is that gene expression occurs as a result of DNA being transcribed to RNA and then translated to proteins. Shortly after the discovery of DNA as the carrier for genetic information and the structure of DNA, this process was dubbed the \emph{Central Dogma of Molecular Biology}~\cite{crick1958protein, macleod1944studies, watson1953structure}. During the first stage of transcription, proteins called transcription factors (TFs) bind to regulatory regions surrounding the gene to recruit RNA polymerase; an enzyme responsible for the synthesis of a sub type of RNA known as messenger RNA (mRNA). The second phase, known as translation, is primarily carried out by a molecule called a ribosome and by translational RNA (tRNA) outside of the nucleus. The end result is a protein comprised of an amino acid chain.
        
        % Changes in gene expression can give rise to different outcomes. 
        There are typically two ways that gene expression can be altered, giving rise to different outcomes; the first being a direct alteration of the genetic sequence. A mutation in the DNA sequence of a gene gene can result in the organism having a different genotype.        % A type of mutation known as a single nucleotide polymorphism, that is, the substitution of a single nucleotide base for another.
        Sickle cell anaemia is the result of the mutation of just a single nucleotide and affects the haemoglobin carrying capacity of blood cells, causing the sufferer to develop an array of health complications such as strokes~\cite{clancy2008dna}. This disease is an extreme case of what can go wrong when mutations occur in DNA. Mutations in organisms that lead to expression of genotypes that are beneficial for an organism can be positively selected for in an organism in an appropriate context. Whether a mutation that resulting in a dfferent genotype is beneficial or not for an organism is context dependent, but this is essentially the corner stone of evolution. Fortunately, there is typically some redundancy when it comes to the genotype of an organism and mutations considered beneficial are often selected for.
        
        The other mechanism/explanation for changes in gene expression is through epigenetic factors; by definition, epigenetic changes are brought about without changes to the DNA sequence itself. There are many more epigenetic factors involved in the process of regulation, such as RNA, acetylation, phosphorylation and more~\cite{geiman2002chromatin, jaenisch2003epigenetic,holoch2015rna, waterland2003transposable}. Each play important roles in the grand scheme of regulating gene expression. For the purpose of this report, the way chromatin remodelling, methylation and certain TFs interact will be the examples discussed here.


        Regulation at the transcriptional level requires the orchestrated efforts of a diverse range of factors, proteins and enzymes with a variety of roles. Among these, a notable example are TFs, as the initiation of the transcription and the recruitment of other factors depends on interactions between TFs and DNA~\cite{lemon2000orchestrated}. The modification of the way these factors interact can affect the final expression profile. 
        
        Transcription factor binding sites (TFBSs) are often described as \emph{cis}-regulatory elements (CREs), since they are often found in \emph{cis} or on the same chromosome that their target gene can be found. These TFBSs are then classified according to their function; either as silencers, promoters, insulators and enhancers~\cite{gaszner2006insulators, gross1988nuclease, li1999locus}. Enhancers are segments of DNA, usually a few 100 base pairs in length, that act as platforms for the recruitment of TFs for the regulation of transcription~\cite{spitz2012transcription}. Regardless of the type of CRE, the main mechanism by which they carry out their function is through TF binding to degenerate/non-coding sequences in the DNA. Spitz and Furlong~\cite{spitz2012transcription} provide a review of the importance of enhancers and how their interactions with each other to form enhancer complexes and different models. 
        
        In order for these TFs to bind to DNA, they require access to their target binding sites. As mentioned earlier, DNA is packaged into chromatin. Depending on the configuration of the packaging, this may hinder or facilitate the expression of a gene by either limiting or increasing the accessibility to regions of DNA~\cite{li2007role}. Chromatin remodelling is the process of repackaging the chromatin and changing the accessibility to different regions in the DNA. For examples, genes located in closed chromatin can be silenced due to the binding sites required for the intiation of transcription not being able to be reached.

        
        The framework that I propose will be one where we use methylation data, collected by one of of our collaborators, to infer the state of chromatin and subsequently, the motifs of sequences that area candidates for TFBSs that we then compare in an alignment free algorithm.

        
        
        % During transcription, TFs are able to recognize short sequences in non-coding regions of DNA and bind to these sites.
        % Upon binding, TFs recruit other TFs to bind to other sites on the DNA or to the proteins themselves.
        % The exact mechanism of how these TF complexes work is not understood very well. 
        % One plausible explanation is that some combination of TFs need to be succesfuly bound to DNA to carry out their function.
        % Another theory suggests that TFs can bind in a non-specific way to DNA sequences as well as other TFs, introducing 
        % a dimension of protein-protein interaction in to the activation of the reglatory element. 
        % There are other suggestions on how TFs might bind cooperatively in order to become active regulatory elements; for a review, 
        % see Spitz and Furlong (2012)~\cite{spitz2012transcription} for a more in depth explanation. 
        
        % For example, the inhibition of a certain protein required for the initiation of a gene can cause a gene to fail to be expressed. Biogists have used this idea as the basis for protocols used in labs to help design experiments \textbf{THERE SHOULD BE A REFERENCE HERE}. The remainder of this section will be devoted to a more focused explanation of epigenetic factors for gene regulation, relating specific elements to the data provided by collaborators and giving context to the project. 
        % Thus, gene expression is regulated in many stages by a variety of factors. Here, the focus is placed on regulation in the context of transcription.
        
% The main mechanism by which TFs interact with DNA, and subsequently regulate transcription
% . the In this work, we focus on transcriptional regulation, through enhancers.

        


% Here are the main ways that gene expression might be altered:
% \begin{itemize}
% 	\item Direct changes to DNA sequence. 
	
% 	\item Epigenetic changes. Gene expression is a complex and multifaceted process. Regulation in the expression
% 		of a gene happens at different levels, at different stages, by different factors. For the purposes
% 		of this work, a focus will be placed on regulation at the transcriptional level. Specifically, the focuse
% 		will be placed on a subset of TFs. We would like to identify and classify these enhancers at a stage
% 		at a sequence level.
% \end{itemize}
        \textbf{What is an enhancer}
        
        % TF binding sites (TFBS) are divided into categories according to the function associated with the TFs that bind to that site.
        % We would like to focus on the category known as Enhancers. Enhancers are a type of TFBS that enhance or upregulate the 
        % expression of a gene associated with that site.
        % To compount things a single enhancer may act on multiple genes at some distance away from the gene of interest. 


        % ?Changes in gene expression can often be attributed to changes in either the direct genetic sequence (\emph{eg.} mutation), or epigenetic factors in the expression process. There are numerous ways that epigenetic factors can alter the expression of genes without changes to the genetic code itself. A readily available example of an epigenetic factor are TFs. These TFs take on a diverse set of roles in regulating transcription, ~\cite{lemon2000orchestrated}.
        
        \subsection{Enhancer Identification}
        
        The identification of enhancers regions in DNA has been a subject of intense study since their discovery in the late 70s and is an active field of research.
        Conservation of sequence across species as a single metric has been used for the identification of enhancers. This approach has proven useful in the past, but in the context of cardiac enhancers, relatively few have been found. Sequencing has been used, in conjunction with knowledge about specific binding proteins to identify enhancer regions~\cite{blow2010chip}. This would suggest that specific approaches need to be developed for pathway or cell specific enhancers. 
        
        
        
        To compound the difficulty in the identification of enhancers, they have show conservation of function across a large evolutionary distance, even if the sequences have diverged significantly~\cite{tautz2000evolution}. Essentially, the sequence in an enhancer binding site may differ through mutation, but the ensemble required for the expression and regulation of certain genes is still able to express the gene in ways that are expected. This conservation of function, without conservation of sequence can be a weakness for aligment based comparative genomics techniques. 
        
        
        
        
        That doesn't mean that there is no loss of sequence conservation. It has been shown in tissue specific studies that gene expression is much higher conserved than TF binding. Positive correlation with combinatorial TF binding and transctipion levels of closeby genes~\cite{wong2014decoupling}
        
        We have some data in the form of ChIP-Seq h3K4me1. This denotes the methylation status of a histone. This particular site often associated with promoter regions of actively being transcribed genes~\cite{barski2007high}. 
        %https://epigenie.com/key-epigenetic-players/histone-proteins-and-modifications/histone-h3k4/
        While this association has been used to identify active and primed enhancers, it is poorly understood whether H3K4me1 influences, is influenced by or correlates enhancer activity~\cite{rada2018h3k4me1}
        
        The many faces of histone lysine methylation 2002)
        
        Revisit Spitz and Furlong to see what sort of idease we have to give an overlay of what a TF is and how we have different models for regulation. 
        
       
        
        
        
        As an example, the remodelling of chromatin through histone methylation can alter the accessibility of transcriptional machinery to reach their target genes~\cite{gibney2010epigenetics, holoch2015rna}. 
        
        % Changes in gene expression is primarily driven by direct changes in the sequence (genetic) or changes in epigenetic factors involved in the expression process, such as chrmomatin remodelling through histone methylation altering accessibility between transcriptional machinery and their target gene~\cite{gibney2010epigenetics, holoch2015rna }.% (RNA CAN ALSO REGULATE GENE EXPRESSION MAYBE) 
        Epigenetics is important~\cite{holliday2006epigenetics}
        DNA Methylation is super important when considering stability of gene epxression~\cite{jaenisch2003epigenetic}
        
        Comparative genomics might not be the best way since these things display modest or no conservation. Sequence conservation does not provide clues to function, even if it is identified as an enhancer~\cite{pennacchio2013enhancers}
        
        
        We have many different 
        \subsection{How are they found in the wet lab?}
            
        \subsection{How are they found computationally?}
        
        
        %     So, transcription factors are proteins that can bind to DNA. These proteins are responsible for the initiation and regulation of transcription. Something about gene control comes from regulation. 
            
        %     To speak some of the lingo, these \emph{cis}-regulatory elements can hang out together and create these \emph{modules} that act as the high density binding sites for these elements. Enhancers are segments of DNA, typically a few 100 base pairs in length that act as binding sites for transcription factors that stimulate the transcription of genes. 
            
        %     Enhancer regions are part of the subset of DNA that is considered to be non-coding. 
        %     These regions can be sites for transcription factor binding and are thought to play an important role in stimulating gene expressions. 
        %     {\color{red} The role of enhancers might vary at different developmental stages}
        %     There are competing models proposed about how these enhancer regions work in the recruitment and attraction of transcription factors and how they encourage binding; these are the billboard, the enhanceosome and the TF collective. While they vary in the mechanisms in the activation of the enhancer, they all rely on a binding mechanism for activation.
            
        %     Part of the difficulty in attempting to characterize these enhancer regions is that unlike coding DNA regions, there is no clear pattern underlying these. There have been previous attempts to identify and classify these elements grounded in different bioinformatic, statistical and mathematical frameworks.
            
        %     The most familiar of these would be sequence segmentation and alignment based approaches.
            
        %     In a pairwise sequence alignment, the algorithm requires two input sequences. What you do 

        \subsection{Computational Identification of Enhancers}
        \label{sSec:computational}
        
        The identification of enhancers regions in DNA has been a subject of intense study since their discovery in the late 1970s and is an active field of research. Traditionally, wet laboratory assays have been used for their identification~\cite{rosenthal198772}. However, bioinformaticians are turning to more mathematical, computational and statistical approaches to address some of the difficulties associated with the characterization of these elements. 
        
        % Each of these methods have their own advantages and disadvantages and will be discussed below.
        There is an increasing amount of data to with the genome and researchers are striving to utilize these in their methods. Enhancer identification typically occurs in two steps; the consolidation of different data types, and using computational methods to annotate DNA regions~\cite{kleftogiannis2015progress}. The computational methods of use are typically dependant on the type of data available and rely on a variety of underlying mathematics. A selection of these methods will be discussed below.
        % With the increasing amount of available data that describes DNA, the modern practice of computational identification for enhancer regions involves the incorporation of various data types. Some example of data types that have been used are conservation \cite{visel2007enhancer}, chromatin profile of histone marks \cite{won2010genome} and chromatin accessibility combined with information about general TFBS \cite{boyle2010high}. Each of these data types provides different information about the state of DNA and will be discussed below. 
        
        % THese data types will be discussed below. 
        
        
        % Sequence based methods rely mainly on statistical techniques. Researchers will input genes and flanking regions that are believed to be coregulated. These statiscal techniques then aim to identify motifs that are statistically over-represented against some background noise. The main difference between these techniques is in the choice of definition of over-representation and what constitutes as noise.

        % Comparative genomics is powerful because it has been used in the past to identify highly conserved genetic regions across species. The downfall comes from the fact that enhancers do not necissarily display the same type of conservation as those of coding regions, but are able to maintain function. In fact, heart tissue is a solid example of where the enhancers are not conserved but maintain function. 

        % S H O W D O W N?

        % How do these methods perform

        % Messaging Mirana if she's back from Maternity Leave. Have someone there who sees things as
        % one of their projects. Collaborator rather than a resource. 

        % MEMEs Ask Mirana about the package that they've developed. 


        % Make a function where the input is a list,
        % the output of that function is a table where each row is the summary of the information in that list 
        % Here are some of the difficulties associated with the identification of enhancers:
        % 	\begin{itemize}
        % 		\item 
        % 		\item
        % 		\item
        % 		\item 
        % 	\end{itemize}
        % What is a comparative method?
        % Give an example of a comparative method
        % What is good about it?
        % What isn't so good about it? this
        A group of computational methods rely on machine learning algorithms and pose the identification of enhancers as a classification problem. Users typically provide a collections of regions flanking genes that are thought to be regulated by similar groups of enhancers. Algorithms are then used to identify DNA sequence motifs that are statistically overrespresented in these regions. One popular suite of programs is the Multiple Expectation Maximization for Motif Elicitation (MEME) suite. The core MEME algorithm fits a mixture model to input a set of input sequences and discovers motifs in the DNA sequence that maximize a likelihood function of the model~\cite{bailey1994fitting}. This class of techniques has an advantage over alignment based ones as the input for the algorithm is a collection of unaligned sequences, so they are able to cope with the combinatorial rearrangement of regulatory regions. 
         
        The next class of techniques rely heavily on the comparison on DNA structure between two species.
        These comparative techniques that were originally developed for the identification of functional coding regions and have been adapted to the identification of TFBSs like enhancers. At the core of these methods is sequence conservation; the idea that functional sequences are under some type of evolutionary pressure to conserve themselves and maintain function. These methods have been used with success in the prediction of functional non-coding elements using large-scale genomic comparisons~\cite{aparicio2002whole, gottgens2000analysis, loots2000identification, mouse2002initial}. An example of one is described below.

       A popular comparative based approach is sequence segmentation. This method seeks to delineate the DNA by some feature and classify the segments according to properties of interest.
            % Someone seriously decided to name their method LASAGNA and it makes use of a database called ORegAnno. The algorithm work on the assumption that TFBS contain a short, highly conserved stretch of DNA~\cite{lee2013lasagna}.
        % Example of such a method is %LASAGNA~\cite{lee2013lasagna}% 
        Example of such a method is Change Point~\cite{keith2006segmenting}. To attempt to overcome some of the weaknesses of comparative techniques that rely solely on conservation, Change Point also utilizes information about the frequency of nucleotide bases associated with function and transition/transvertion ratio. The program encodes a DNA alignment between different species into a string representing the information discussed. A Bayesian hierarchical model is used to obtain distributions for the positions of boundaries for segments in the string and classify them into distinct groups with homogeneous properties. Since the inception of the program, it has been used to investigate genomic structure and for the identification of TFBS~\cite{algama2014investigating, algama2017genome}. 
        
         A complication associated with some alignment based techniques for the identification of enhancers is the observed conservation of function across a large evolutionary distance, even if the sequences have diverged significantly~\cite{tautz2000evolution, pennacchio2013enhancers}. While it has been shown that TFBS show significantly higher conservation when their target gene is expressed in both species, it is not necessarily true that sequence conservation provide clues to function, even if an element is identified as an enhancer~\cite{hemberg2011conservation, pennacchio2013enhancers}.
         The sequence in an enhancer binding site may differ through mutation or the combinatorial rearrangement of elements, but the ensemble required for the expression and regulation of genes is still able to carry out it's expected function~\cite{wong2014decoupling}. This conservation of function, without conservation of sequence can be a difficulty for alignment based, comparative genomics techniques. 
        % This program takes an alignment and produces an encoded string that represents various information such as: conservation, CG content, transition/transvertion ratio. 

        % It then segments the string and allocates the segments into distinct groups with homogeneous properties. 

        % Now talk about epigenetic
        % ones and sequence based ones. Common downfalls. Also HEART STUFF
        Epigenetic markers are increasingly being used in computational identification. Villar et al.~\cite{villar2015enhancer} used a combination of methylation and acetylation data to identify and study enhancer regions and promoter sequences in a genome wide study; they concluded that in mammalian species recently evolved enhancers are a the dominant feature in the regulatory landscape. Depending on the type of epigenetic data available, researchers are able to incorporate information about the features of the chromatin landscape, histone methylation and active enhancers. 
        
        
        ChromHMM is a program based on the Hidden Markov Model (HMM). 
        
        ChromHMM notes~\cite{ernst2010discovery}:
            \begin{itemize}
                \item Uses HMM
                \item Input is epigenetic markers (potentially lots of them)
                \item Output is different ``states" associated with different elements. Promoters, intergentc stuff, repressed.
            \end{itemize}
       
       A more technical explanation of HMMs see Reference
        
        % A variety of these techniques have been developed in this vein for the identification of TFBS~\cite{tompa2005assessing}. 
        
        
        
        
        % However, the results of such computational tools still need additional work to be classified as enhancers. 
        
        A conservation based approach has proven useful in the past when identifying TFBSs in comparative techniques, but in the context of enhancers relatively few have been found. Although conservation of sequence across species as a single metric for enhancer identification has been used in the past, it alone is not sufficient to identify enhancer activity~\cite{blow2010chip}. Instead, it is typical to incorporate more information about the cellular system or pathway of interest. 
        % Sequencing has been used, in conjunction with knowledge about specific binding proteins to identify regions of interest. 
        This would suggest that specific approaches need to be developed for pathway or cell specific enhancers. 

        
        
        % However, some of the difficulties associated with epigenetic marks are as follows. Often these data are very cellular or pathway specific and thus any results obtained are tissue specific. 

        % That doesn't mean that there is no loss of sequence c /onservation. 
        Some of the problems associated with the characterization of enhancer regions are as follows:
            \begin{itemize}
        		\item Many enhancers are active at a given time in any given tissue. 
        		\item There is a many-to-many correspondence between enhancers and genes. That is, a single enhancer may be acting on multiple genes and each gene may have multiple enhancers acting on it. 
        		\item Traditional wet-lab assays can can find only one enhancer at a time. 
        	\end{itemize}
        	
        	
        We have some data in the form of chromatin immunoprecipitaion followed by sequencing (ChIP-Seq) about the monomethylation of histone H3 at lysine 4 (H3K4me1). This type of methylation has been used to identify active and primed enhancers, but it is poorly understood whether H3K4me1 influences, is influenced by or correlates enhancer activity~\cite{rada2018h3k4me1}. This methylation of this particular site often associated with promoter regions of actively being transcribed genes~\cite{barski2007high}. 
         
        \subsection{Cardiac Enhancers}
        

In this thesis, our interest lies in the identification and classification of certain regions in the DNA that help regulate the expression of genes in cardiac tissue. We seek to identify and classify active enhancers that are involved in the process of heart formation. Cardiac genes are of particular interest because it has been shown that there is reasonable conservation of protein coding genes between different eukaryotic species, yet there is differences in physiological features such as the number of chambers in a heart~\cite{harvey1996nk}. %\textbf{That need to be described below}. 

The classical wet laboratory technique for the identification of enhancers is using a protocol involving Chromatin immunoprecipitation followed by sequencing (ChIP-seq). A summary of the chIP-seq protocol is provided by \cite{collas2010current}:
    \begin{enumerate}
        \item Proteins bound to DNA are given a treatment to solidify their binding;
        \item Cells in a sample are homogenized;
        \item The chromatin is cut into segments;
        \item The segments that have been bound to the protein of interest are selectively separated through immunoprecipitaion;
        \item These DNA fragments are then used as input for sequencing.
    \end{enumerate}
These DNA fragments may be associated with TFBSs depending on the proteins that are being introduced and bound to DNA. The purpose of ChIP-seq is to investigate the interaction between proteins and DNA.

There are variations of ChIP-seq that been used to find active enhancers in the past~\cite{visel2009chip}, but the protocol has its weaknesses. A standard method is the introduction of a protein associated with enhancer binding such as p300 and identifying the regions that it binds to, but this method is only able to identify a subset of active enhancers in a given tissue~\cite{heintzman2007distinct}. Identifying the subset of enhancers that are active in tissue and what genes they may be regulating is an extension of the problem of simply just searching for enhancer regions.

Researchers have modified existing protocols in attempts to overcome some of the problems that exist. These modifications have been used to identify prospective cardiac regulatory sequences distinct from those found by standard protocols~\cite{he2011co}. These researchers investigated combinations of TFBSs to identify enhancers that are active in cardiac tissue with the hypothesis that certain combinations of TFBSs were present in in enhancer regions. This method proved fruitful as they were able to identify distinct putative regions and regions that were previously reported by others~\cite{blow2010chip}.%What these guys also did is they looked at combinations of transcription factors to identify the enhancer regions active in the heart.

Computational methods applied for the identification of cardiac specific enhancers. A conservation based approach has proven useful in the past when identifying TFBSs in comparative techniques, but in the context of enhancers relatively few have been found. Although conservation of sequence across species has been used as a metric to identify enhancer regions, conservation alone is not sufficient to identify enhancer activity~\cite{blow2010chip}. This again highlights some of the drawbacks of relying heavily on conservation. Instead, there is a movement to incorporate more information about the cellular system or pathway of interest. 
 
Researchers interested in cardiac tissues have had success with combining different types of data associated with DNA. They have combined ChIP-seq data for proteins associated with cardiac TFs, computationally predicted enhancers, enhancer associated monomethylation of histone H3 at lysine 4 (H3K4me1) modification H3K4me1 as well as conservation to identify enhancer regions and link change of the regulatory landscape with congenital heart disease ~\cite{smemo2012regulatory}. These researchers have found evidence for a large population of poorly conserved heart enhancers that are still able to function as active enhancers. They have also suggested that the evolutionary constraint on enhancers varies by tissue type and that specific approaches need to be developed for pathway or cell specific enhancers. We hypothesize that despite the apparent lack of sequence conservation, the combinatorial binding of TFs is conserved between species. 

% The use of conservation isn't all that bad. Combined with the p300, a protein commonly associated with enhancers, ChIP-seq assays, are able to predict active enhancer elements in different tissues. However, in the context of cardiac tissue, . It is suggested that the evolutionary constraint on enhancers varies by tissue type~\cite{blow2010chip}. 

       
        % Sequencing has been used, in conjunction with knowledge about specific binding proteins to identify regions of interest. 
        % This would suggest that specific approaches need to be developed for pathway or cell specific enhancers. 


%   We have some data in the form of chromatin immunoprecipitaion followed by sequencing (ChIP-Seq) about the monomethylation of histone H3 at lysine 4 (H3K4me1). This type of methylation has been used to identify active and primed enhancers, but it is poorly understood whether H3K4me1 influences, is influenced by or correlates enhancer activity~\cite{rada2018h3k4me1}. This methylation of this particular site often associated with promoter regions of actively being transcribed genes~\cite{barski2007high}. 
   

        \subsection{Mathematical Techniques}
        \label{sSec:mathematical}
        This section is a more in detailed discussion of the mathematics underlying the computational methods discussed above in Section~\ref{sSec:mathematical}. A focus will be placed on those techniques that are comparative in nature and those that are able to take into account epigenetic markers as input. It is important to mention the MEME family of algorithms as the method is prevalent in the field of TFBS identification, but they will not be discussed here.%They will not be discussed here as they do not take into account available epigenetic information, nor are they comparative. 
        Interested readers are directed to the MEME suite website (\texttt{http://meme-suite.org/}) and Section~8.5-6 of Parida~\cite{parida2007pattern}.

            % \subsubsection{k-Means Clustering}
            % The aim of K-means algorithm is for putting $N$ data points in a space of $I$-dimension into $K$ clusters. Let $S$ denote the space the data is found in. Denote the data points as $\{x^{(n)}\}$ where $n$ runs from $1$ to $N$ and each $x^{(n)}\in S$. Further, let $d$ be a metric that is defined over $S$; $d$ is usually Euclidian. The algorithm is initialized with the setting of $K$ means $\{m^{(k)}\}, usually to random values. The algorithm operates in two steps; an assignment step, and an update step.

In the Assignment step, each data ppoint is assigned to the nearest mean. The guess for cluster $k^{(n)}$ that $x^{(n)}$ belongs to by $\hat{k}^{(n))}$, where
	\begin{equation}
		$\hat{k}^{(n)}$ = \argmin_k\{d(m^{(k)},x^{(n)})\}.
	\end{equation}
In the event of a tie, $\hat{k}^{(n)}$ is set to the smallest $k$.

In the Update step, 

Given a set of means $\{k^{(n)}\}$, each data point $x^{(n)}$ is assigned to a cluster according to  
	\begin{equation}
		\argmin_{k\in K}\{d(m^{(k)},x^{(n)})\}
	\end{equation}
where $K$ is the set of clusters, defined by the mean $m$. Now, let the set of the mean of the data assigned to cluster $k$ be $S_k$.

The update step then consists of updating the means according to
	\begin{equation}
		k_i = \frac{1}{|S_k|}\sum_{x_i \in S_i} x_i
	\end{equation} 

These two steps are iterated over until a stopping criteria is met. Typically, when no data points change clusters, the algorithm is said to have converged.
            

            \subsubsection{Markov Chains}
            While Markov chains are not explicitly mentioned in most of the computational methods described above, a basic treatment of this mathematical object will be beneficial for the reader as some of the underlying techniques below rely on some of the properties of markov chains.

A Markov chain is a sequence $X_{t_1}, X_{t_2}, ...$ of random elements $X_i$ from some state space $\mathscr{X}$ with the following property
%such that the distribution of transition probabilities at the next state of the chain, conditional on all previous states, depends only on the current state of the chain; that is 
    \begin{equation}
        P(X_{t_{n+1}}|X_{t_n},X_{t_{n-1}},\ldots X_{t_1}) = P(X_{t_{n+1}}|X_{t_n});
    \end{equation}
this is known as the Markov property.
% {\color{red}
% {I'm not sure if the states should be upper case or lower case}}. 
%For illustrative purposes, here only discrete time Markov chains on discrete state spaces will be considered. 
% The motivating reason for the inclusion of Markov chains is because of Markov chain Monte Carlo (MCMC) being discussed below.
% The main idea behind MCMC is generating a Markov chain so that the limiting distribution of $X_t$ as $t\rightarrow\infty$ is some $\pi$. 
% In practice, this $\pi$ is the posterior distribution obtained for some quantities of interest in a Bayesian framework. 
% Ensuring that the distribution of $X$ converges to $\pi$, the chain must be constructed with certain properties. However, to describe these mathematically it is necessary to define the probability that the chain moves from state $i$ to state $j$ in $t$ steps as $ P_{ij}(t) = P(X_t = j | X_0 = i)$
% and the first return time to state $i$ to be $\tau_{ii} = \min\{t>0:X_t = i|X_0 = i\}$
%     \begin{enumerate}[(i)]
%         \item a \textbf{positive recurrent} Markov chain is one with the propery
%             \begin{enumerate}
%                 \item $E[\tau_{ii}] < \infty ,\, \forall i$
%             \end{enumerate}
        
%         \item A Markov chain is called \textbf{irreducible} if $\forall~i,j, \exists~t > 0$ such that $P_{ij}(t) > 0$.
%         % \item An irreducible chain is \textbf{recurrent} if $P(\tau_{ii} < \infty) = 1$ for some $i$.
%         % \item An irreducible recurrent chain is called \textbf{positive recurrent} if $E[\tau_{ii}]<\infty$ for some $i$.
%         \item An irreducible chain is \textbf{aperiodic} if for some $i$
%             $$gcd\{n>0:P(X_n = i|X_0 = i) > 0\} =1. $$
%     \end{enumerate}
% The irreducibility condition intuitively means that all states are reachable from any starting state in a finite number of steps, while requiring aperiodicity means that the chain will not oscillate between a subset of states in a periodic manner and positive recurrence guarantees that the chain will not drift to infinity. The combination of the three properties 
            
            \subsubsection{Parameter Estimation}
            \subsubsection{Maximum Likelihood Estimation}
Maximum likelihood estimation is a frequentist statistical technique for infering parameters. Given a density $f(x|\theta)$, and a set of sample values $x_i,\ldots,x_n$, the likelihood fucntion is defined as  
	\begin{equation}
		p(\theta|D) = \prod_{i=1}^{n}f(x_i|\theta)
	\end{equation}
where $D$ denotes the collection of data. The maximum likelihood estimate of the parameter $\theta$, which will be denoted by $\hat{\theta}$ is the value that maximises $L(\theta|D)$. For simple problems analytic solutions may exist. In practice, such solutions are intractable and computational methods used instead.

\subsubsection{Bayesian Estimation}
The advantage using a Bayesian framework is that rather than giving point estimates for quantities of interest, we are working in distributions and are able to give a have a measure of uncertainty. Given some prior distribution (belief) about our parameter $p(\theta)$ and the likelihood of the data, given the parameters $p(D|\theta)$ we can obtain what is known as the posterior distribution $ p(\theta|D)$ of the parameters given the data 
    \begin{equation}
        p(\theta|D) = \frac{p(D|\theta)p(\theta)}{p(D)}
    \end{equation}
In Bayesian inference, the marginal likelihood $p(D)$ is some constant and the rule is often written as 
    \begin{equation}
        p(\theta|D) \propto p(D|\theta)p(\theta).
    \end{equation}
It can be very difficult to sample from the posterior distribution analytically since there may be no closed form for the distribution, except in the most simple toy problems. In practice, Markov chain Monte Carlo (MCMC) is a standard technique used to sample from these distributions. 

The basis of MCMC is taking a constructing a sequence of stochastic processes $$\{X_0,X_1,\ldots,X_n\}$$ with the property $p(X_k|X_{k-1},\ldots X_0) = p(X_k|X_{k-1})$ in such a way that the distribution $p(X)$ limits to some $\pi$. In the context of Bayesian inference, this $\pi$ is the posterior distribution that is desired to be sampled from.
            
            \subsubsection{Markov Chain Monte Carlo}
            
	The primary purpose of Markov chain Monte Carlo (MCMC) in the context of Bayesian inference is to sample from posterior distributions for which more efficient means of sampling are not available.
	%obtained in doing Bayesian analysis. 
	{\color{red}This is achieved by constructing a Markov Chain that is aperiodic and irreducible, (therefore ergodic) to ensure that the limiting distribution of the Markov chain is the the one that is desired to be sampled from. This method guarantees that $X_t\rightarrow\pi$ in in distribution $t\rightarrow\infty$}
% 	{\color{red}Apparently we only ever make Markov Chains that are reversible. I think this might be so that the detailed balance equations hold, which is sufficient condition for ergodicity(?)}
% 	\textbf{Practical considerations:} Mixing, convergence, proposal distributions, how long to run the chain. 
	\subsection{Metropolis-Hastings Algorithm}
	The Metropolis-Hastings (MH) Algorithm is the standard introduction to MCMC methods for sampling. Presented in 1970 by Hastings \cite{hastings1970monte}, it is an extension of the algorithm originally put forth by Metropolis in 1953 \cite{metropolis1953equation}.
	The steps for the algorithm, also presented in Algorithm~\ref{alg:MetHast}, are as follows. Let $\pi(x)$ be the target density (up to a constant), defined on some set ${\mathscr X}$.
	To initialize the algorithm let $x_0$ to be the first sample, chosen arbitrarily. In each iteration $g(y|x)$ 
% 	({\color{red}Is it more conventional to write $g(Y|X)$ or does it not matter as long as it is consistent?}), 
    known as the {\color{red}transition kernel} is sampled from to generate a candidate point. The candidate point $y$ is accepted with probability 
	    \begin{equation}
	        \alpha(x,y) = min\left(1,\frac{\pi(y)g(x|y)}{\pi(x)g(y|x)}\right).
	    \end{equation}
	If the candidate point is accepted, $x_{t+1} = y$, otherwise, $x_{t+1} = x_t$ and the next iteration is begun. In the original Metropolis algorithm, $g$ is chosen to be symmetric; that is $g(y|x) = g(x|y)$. However, in general $g$ can be arbitrary. There are smart ways to tune $g$ to ensure that the chain mixes adequately.
	
	\begin{algorithm}[H]
	\label{alg:MetHast}
 %\KwData{this text}
 %\KwResult{how to write algorithm with \LaTeX2e }
 Set $x_0$, $t=0$\; 
 \Repeat{converged}{
  Draw $y\sim g(y|x)$\;
  Draw $u\sim U(0,1)$\;
  \eIf{$u<\alpha(x,y)$}{
   $x_{t+1} = y$\;
   }{
    $x_{t+1} = x_t$\;
  }
  Increment $t$\;
 }
% 		\begin{enumerate}
% 			\item Generate a candidate point $x'$ from $g(x'|x_0)$.
% 			\item Calculate the acceptance probability
% 				\begin{equation}
% 				\alpha = \frac{f(x')}{f(x_t)}
% 				\end{equation}
% 				and generate $r\sim U(0,1)$ and compare with $\alpha$ to decide whether or not to accept the candidate point probabilistically.
% 			\item Repeat steps 1. and 2.
% 		\end{enumerate}
% 	\subsection{Gibbs Sampler}
% 	The Gibbs sampler is presented as a method for sampling from some distribution over a set ${\mathscr X}$ with dimension $n$. Let $x = (x_1,...,x_n)$ and let us denote the $i^{th}$ sample as $x^{(i)} = (x_1^{(i)},...,x_n^{(i)})$.


 \caption{Metropolis-Hastings Algorithm}
\end{algorithm}

\subsection{Gibbs Sampler}
In general, the quantity $x$ is a vector and standard MH updates all the components at once. In contrast to this, there exist a class of MCMC methods known as Gibbs samplers, that iteratively update the components of $x$. The Gibbs Sampler originally presented by  Geman and Geman \cite{geman1984stochastic} is an example of one of these and can be shown to be a special case of single component MH \cite{gilks1995markov}. 
% Rather than updating the entirety of $x$ as in the MH Algorithm, there exist a class of MCMC algorithms that update components of $x$ in an iterative fashion, the most popular of these being the Gibbs Sampler, originally formulated by Geman and Geman \cite{geman1984stochastic}. 
Denote $x_. = (x_{.,1},\ldots x_{.,h})$. Note that each of the components of $x$ may possibly be of differing dimension. 
% ({\color{red} I suspect that this has something to do with efficiency. I.e. you might want to group components together that have 'nice' conditional distributions?}).
Introducing the notation $x_{.,-i} = (x_{.,-i},\ldots x_{.,i-1}, x_{.,i+1} \ldots x_{.,h})$, that is $x_{.,-i}$ comprises of all the components of $x_.$ except $x_{.,i}$. Gibbs sampling updates $x_{.,i}$ by drawing a candidate $y_i$ from $\pi(y_{.,i}|x_{.,-i})$, also known as the full conditional distribution. An advantage of the Gibbs Sampler is that all candidate points are accepted with probability 1.

\begin{algorithm}[H]
 %\KwData{this text}
 %\KwResult{how to write algorithm with \LaTeX2e }
 Set $x_0$, $t=0$\; 
 \Repeat{converged}{
    \For{$i$ = 1 to $h$}{
  Draw $y\sim \pi(y_{t,i}|x_{t,-i})$\;
  Set $x_{t,i} = y$
  }
  Set $x_{t+1} = x_t$\;
  Increment $t$\;
 }
  \caption{Gibbs Sampling Algorithm}
\end{algorithm}
            
            % \subsubsection{MEME}
            % Different computational tools exist for the identification of novel regulatory elements. In practice, a user p

MEME requires a number of sequences that are thought to be coregulated. This knowledge is obtained through protein assays that are done in a laboratory. Users want to obtain learn a motif and determine where each motif begins in each of the sequences given.  
            \subsubsection{Hidden Markov Models}
            The following is a brief description of hidden Markov Models (HMMs) adapted from \cite{mesa2016hidden}.

Let $x = (x_1,x_2,\ldots,x_n)$ be a sequence of observations where each of the residues $x_j$ takes values from a finite alphabet $D$. 
Also $y = (y_1,y_2,\ldots,y_n)$ is a collection of states, where each $y_i$ belongs to $\mathcal{E}$, a finite set of states. The sequence $x$ can be thought of as a time series where the indeces $j$ are taken as discrete time steps. These models are considered 'hidden' because only sequence of residues $x$ is observable and but the sequence of states is unkown.

The model starts at some initial state, produces a residue and moves to a new state and procudes a new residue. This continues until the end of the sequence. Let $P(x_i|y_i = k)$ denote the probability that the emission $x_i$ was produced by state $k$ at time $i$. Next, let the probability that a transition is made from state $k$ to state $l$ given by $P(y_{i+1} = l | y_i = k)$. The joint distribution for the probability the model transits path $y$ and generates the sequence $x$ is then
	\begin{equation}
		P(x,y) = \prod_{i=1}^{n} P(y_i|y_{i-1})P(x_i|y_i),
	\end{equation}
where $P(y_i|y_{i-1}) = P(y_{i} = l | y_{i-1} = k)$ and $P(x_i|y_i) = P(x_i|y_i = k)$. 

The parameters of the model are the transisition probabilities and the emission probabilities. In practice, these parameters are estimated, for example with Expectation Maximization.

There are three assumptions underlying the model and they are:
	\begin{itemize}
		\item The Markov property holds for the states $y_i$. That is $P(y_{i+1}|y_i,\ldots,y_1) = P(y_i|y_{i-1})$.
		\item The Markov chain produced by the states $y$ is homogeneous. That is the probability of transitioning from one state to another is independent of time $i$. 
		\item Conditional independence of observations. That is $x_i$ is only dependant on the state $y_i$. 
	\end{itemize}

An example of usage would be to assign the alphabet $D = \{A,C,G,T\}$ representing the four nucleotide bases found in DNA and the states as $\mathcal{E} = \{\text{intron}, \text{exon}, \text{intergenic region}\}$. These models have been used in computational biology \textbf{cite?}

%albus, silas, salus, meris, janus, miles, myles, regis, remus  





 
            
            \subsubsection{Sequence Alignment}
            % Aligning of sequences can be posed as a mathematical problem. Consider two sequences $S_1$ and $S_2$ (possibly of different lengths) with residues from some finite alphabet $D$. Let one of the sequences (usually the longer one) be the \textit{reference sequence}; this is the sequence that the other one will be aligned to. The typical alignment approach considers the edit distance between the two sequences. This is achieved by either inserting, deleting or substituting characters in certain positions in the sequence to be aligned and assigning penalties to these operations, while scoring matching positions. An alignment is obtained through using a scoring scheme and optimizing an objective function. 

Aligning of sequences can be posed as a mathematical problem. Consider two sequences $S_1$ and $S_2$ (possibly of different lengths) with residues from some finite alphabet $D$. Let one of the sequences (usually the longer one) be the \textit{reference sequence}; this is the sequence that the other one will be aligned to.
A standard implementation of pairwise sequence alignment is obtained by considering the edit distance between two sequences and optimizing a scoring function. The edit distance is obtained by considering the number of insertions, deletions and substitutions of residues in the sequence to be aligned relative to the reference sequence. To achieve an optimal pairwise alignment the two sequences are stored in a table with the residues in columns and a (possibly negative) score is assigned to each column. Typically, columns where insertions, deletions or substitutions have taken place incur a penalty, while matching sequence entries have an associated positive score. The scoring system plays a large role in the generation of the alignment. As such, at least in the field of bioinformatics there are widely accepted scoring schemes such as PAM or BLOSSUM~\cite{mount2008comparison}. The selection of these scoring matrices typically depends of the application.

A variety of algorithms exist for the purpose of multiple sequence alignment (MSA) as opposed to just a pair of them. The class of algorithms associated with alignment problems, lends itself to the style of dynamic programming. Finding a pairwise alignment of two sequences is typically obtained by inserting gaps into the shorter sequence that maximizes the similarities of the input sequences according to some scoring matrix~\cite{edgar2006multiple}. One of the standard approaches for constructing multiple sequence alignments for $N$ sequences is to do $N-1$ pairwise alignments of pairs of sequences, under guidance from phylogenetic trees~\cite{feng1987progressive}.
	The alignment of nucleotide or protein sequences is useful because it can give insight into the different mutations that have occurred through genetics that give rise to different phenotypes. 


The process described above it known as global sequence alignment. By attempting to optimize the overall alignment of the two sequences, long regions of low similarity may be included~\cite{needleman1970general}. In contrast, local alignments seek similarity between subsequences that are somewhat conserved; this approach also has the affect of unconserved resiongs not  contributing to the similarity measure~\cite{goad1982pattern, sellers1984pattern, smith1981comparison}. As with global sequence alignment, there exist scoring matrices; the PAM and BLOSSUM matrices have variants for local alignment~\cite{mount2008comparison}. 
% 	\subsection{Genetic Segmentation}
% 	The act of segmenting a genetic sequence according to some criterion is known as genetic segmentation. There is some debate as to whether or not the genome actually follows a segmented structure {\color{red}CITE}. 
% 	% 
% PAM OR BLOSUM \cite{mount2008comparison}
            
            \subsubsection{Change Point Detection}
            Change point detection is a statistical problem concerned with determining when a change has occurred in a stochastic process underlying data generation. 
Consider a game with 2 $D$-sided dice, each with a different probability distribution governed by a parameter $\theta$ such that $\theta_1 \neq \theta_2$. A dice is rolled an unknown number of times and then changed for the another. 
%This process is repeated $\kappa \leq k_{max}$ number of times, with $\k_{max}$ unknown. 
A record of each of the and all the outcomes are recorded. Given only the concatenation of all the outcomes, the goal of change point detection is to determine when the dice were changed and the distribution of each die. 
% Consider a game where there are two coins with probability $\theta_1$ and $\theta_2$ of producing heads, with $\theta_1 \neq \theta_2$. One of the coins is flipped and unknown number of times and exchanged for the other, which is also flipped an unknown number of times. 
% The sequence of the flips is recorded in sequence and this data is all that is known, as well as the number of coins.
% In this example, the goal of change detection is to determine at what point the coins were exchanged as well as the estimation for the parameters $\theta_1$ and $\theta_2$. 

% In bioinformatics, it common for the number of change points to be unknown {\color{red}perhaps corresponding to the start and end of putative functional elements}; this is a typical extension of change point detection. 
To generalize the above problem, the data of primary concern are sequences of amino acids or nucleotide bases with an unknown number of segments. 
Instead of only dealing with two dice, there may be $k_{max}$ $D$-sided dice (with outcomes $\{1,\ldots,D\}$) having different biases (i.e that $\theta_i\neq \theta_j$ for $i\neq j$). 
The $i$th die is rolled $C_i$ times until unknown number of dice $\kappa$ are used, with $\kappa\leq k_{max}$. In this model, each die is used at most one time and each segment is assumed to be generated by a different die. The total length of the sequence is will be given by, 
    \begin{equation}
        J = \sum_{i=1}^{\kappa}C_i.
    \end{equation}
Given only the concatenation of the outcomes $S = s_1s_2\ldots s_J$, where $s_i$ is the outcome produced at position $i$, the goal of the change point problem is to be able to determine the points where the die have been switched. In the context of biopolymer molecules the sequences of interest may be nucleotides $s_i \in \{A,C,G,T\}$.
% {\color{red}SOMETHING ABOUT SLIDING WINDOW. One approach to a change dection problem is the use of a sliding window analysis that take a sldkfasldkfjas;dlfkj something something. It is sensitive to window size. Noise. etc}

The following is an outline of a Bayesian method for detecting change points, originally presented by Liu and Lawrence \cite{liu1999bayesian}. We define the change points $A_k$ to occur the first time a new die is used. That is
    \begin{equation}
        A_k = \sum_{\nu=0}^{k}C_\nu + 1
    \end{equation}
where $C_\nu$ is the number of times the $\nu$th dice was rolled, $C_0 = 0$ and $k = 1,\ldots,\kappa$. The quantities of interest are the number of change points $\kappa$ and their locations $\*A = (A_1,\ldots,A_\kappa)$. In this framework, $\kappa$ is chosen through model selection as the number of parameters changes with this parameter. Continuing with the dice analogy, let $\Theta_i = (\theta_{k1},\ldots,\theta_{iD})$ represent the distribution of the dice used in the $i$th segment; that is $P(s_j = d) = \theta_{id}$ in segment $i$. With the assumptions that the segments are independent of one another given the change point positions, the likelihood for the segments and change point positions can be written as the product of the likelihood for each segment and the prior for the change point positions given $\kappa$;
    \begin{equation}
        P(S,A|\Theta, \kappa) = \prod_{i=1}^{\kappa} P(S_{[A_{i-1},A_{i}}|\Theta_i)P(A|\kappa).
    \end{equation}
In this equation, $S_{[i,j)} = s_i\ldots s_{j-1}$. The joint distribution can then be written as
    \begin{equation}
        P(S,A,\Theta,\kappa) =  P(S,A|\Theta, \kappa)P(\kappa)P(\Theta)
    \end{equation}
under the a priori assumption that $\kapa$ and $\Theta$ are independent. From here, marginal distributions for the quantities of interest can be obtained and sampled from using MCMC, such as Gibbs sampling as in Section~\ref{ssSec:GS}. 

Some limitation is in the model selection of the number of change points $\kappa$ and the idea that each segment as generated by different processes. 
Change Point by Keith~\cite{keith2006segmenting} is an extension of the model incorporating the comparative techniques from sequence alignment, including $\kappa$ into the inference as well as introducing the idea of allocating segments to classes of homogeneous properties, where non-adjacent segments may have been generated by the same underlying process~\cite{oldmeadow2009multiple}. The posterior distributions obtained in Change Point are on state spaces with differing dimension due to the inclusion of $\kappa$ into the inference. As such, conventional MCMC algorithms like MH and Gibbs will not work to sample from these posteriors, but the GGS in Section~\ref{ssSec:GS} works fine for such a situation~\cite{keith2004generalized}.
% {\color{red} Work on a general framework on how sequence segmentation wotks in Bayesian}

% Consider a sequence $A = A_1A_2,\ledots$

% Sequence change detection is the 
            
            
            \subsubsection{Alignment Free Sequence Comparison}
            Alignment Free stuff:

Alignment free sequence comparison overcomes some of the shortcomings of alignmenet based sequence comparison. The primary concern is when there is a comparison of sequences of regulatory elements taht are note conserved in a traditional alignment sense. This occurs when the sequences being compared are not orthologous (derived from a common ancestor). 

An approach to alignment free sequence comparison is by fixing a word length $k$, and computing the distributions for the frequencies of all $k$-length words (also '$k$-mer') in each of the sequences. 

\section{Math}

let $\mathbf{A} = A_1A_2\ldots A_{n_1}$ and $\mathbf{B}=B_1B_2\ldots B_{n_2}$ be two sequences of length $n_1$ and $n_2$ respectively. Each character $A_i,\,B_i$ take values in some finite alphabet $D$, with $|D| = d$.  The sequences $\mathbf{A}$ and $\mathbf{B}$ are assumed to be made by some $\omega th$ order markov process; $\omega$ may be different for each sequence. 

For convenience, define $\bar{n_1} = n_1 - k + 1$ and $\bar{n_2} = n_2 - k + 1$
            
            
            
            
    % \section{Conceptual Framework}
        
    % \section{Preliminary Data}
    \section{Relevance of study}
        There is some mystery surrounding cardiac enhancers as they are not conserved in the same way that enhancers in other tissue may be, yet are still able to retain function. Being able to rank the importance of enhancers has been done in other tissue in the past by using conservation as a metric, but a lack of conservation for cardiac enhancers poses a problem.
        The aim of the project is to extend existing methods for the computational identification of conserved developmental cardiac enhancers or the development of new ones that are able to rank the importance of enhancers. The the extent that the literature was reviewed, there appears to be a gap in computational methods that function this way in cardiac specific tissue.
        
        We have access to ChIP-Seq H3K4me1 data collected by collaborators for different species in heart tissue in conjunction with those that are available in public databases, at a variety of developmental stages. This type of methylation has been used to identify active and primed enhancers, but it is poorly understood whether H3K4me1 influences, is influenced by or correlates enhancer activity~\cite{rada2018h3k4me1}. This methylation of this particular site often associated with promoter regions of actively being transcribed genes~\cite{barski2007high}. We would like to develop methods that are able to use this data for the computational identification of heart enhancers or extend existing methods to include this data. Whatever method gets developed, it is aimed to be published in the public domain to be used as a tool for bioinformatics and applicable in different tissues specific contexts.  
        
    \section{Progress to date}
    The last 12 months have been spent learning the background biology, the computing languages; Python, R and C++
A variety of scripts have been written in different languages to help in the process of model selection and plotting of 
    
    \section{Work to be done}
    % \begin{figure}[H]
    %     \centering
    %     \noindent\begin{tabular}{p{0.17\textwidth}*{20}{|p{0.01\textwidth}}|}
% The top line
\textbf{Gantt chart} & \multicolumn{4}{c|}{Year 1} 
           & \multicolumn{4}{c|}{Year 2} 
           & \multicolumn{4}{c|}{Year 3} 
           & \multicolumn{4}{c|}{Year 4} 
           & \multicolumn{4}{c|}{Year 5} \\
% The second line, with its five years of four quarters
\rpt[5]{& 1 & 2 & 3 & 4} \\
\hline
% using the on macro to fill in twenty cells as `on'
PI        \on[20] \\
\hline
PDoc 1    \on[20] \\
\hline
PDoc 2    \on[20]  \\
\hline
% using the on macro followed by the off macro
PhD 1     \on[16] \off[4]\\
\hline
% The mbox prevent packages from being hyphenated
% The multicolumn produces no vertical guides within the columns it spans, but
% does put one at the end to complete the righ-hand edge of the table
\textbf{Work \mbox{packages}} & \multicolumn{20}{c|}{} \\
\hline
Finding Bugs  \on[2] \off[6] \on[2] \off[10] \\
\hline
Squashing Bugs \off[2] \on[4] \off[4] \on[1] \off[9] \\
\hline
% Note the omitting the count to on or off is the same as setting the count to 1
Producing Results \off[6] \on[13] \off \\
\hline
Dissemination \off[19] \on \\
\hline
\end{tabular}
    %     \caption{Caption}
    %     \label{fig:my_label}
    % \end{figure}
    
    \subsection{Alignment Based Enhancer Identification}

Change Point has been used in the past to identify non-coding putative functional elements (ncPFEs) that have been conserved across species. We seek a way to extend the method to incorporate data made available to us and further classify these elements according to their function. Specifically, we desire to be able to identify enhancer elements and, if possible, distinguish those that are tissue specific.

Using an unpublished manuscript started by a previous PhD candidate and their data, I will attempt to recreate the results they obtained; a set of ncPFEs. Once these ncPFEs are obtained, I will then attempt to classify the elements by using a combination of binary classifiers about features associated enhancer regions in an application of the idea presented by Keith and Boyd~\cite{keith2012bayesian}. I intend to incorporate the methylation data provided by our collaborators into the binary classification. The results of this analysis will then be compared against existing databases of known elements to evaluate the efficacy of this approach. 

One of the longer term goals of this project is to publish the scripts needed for the model selection, analysis and visualization for Change Point and binary classification in the computing language R as a package with an accompanying publication.

% Then, I plan on applying using our collaborator's methylation data in the form of a binary classifiers combined withite{keith2012bayesian}

\subsection{Alignment Free Enhancer Identification}

In the literature reviewed there does not appear to be an alignment free method that for the identification of enhancers that incorporates epigenetic markers, such as methylation data. For the second project, I would like to investigate the viability of this method on the genome scale. My proposed modification to the alignment free would be a weighting in the calculation for comparison statistic used e.g. weighting the terms on the RHS of Equation~\ref{eqn:AF} depending on whether the region they are located in is methylated or not. This may have the affect of biasing the the sequence comparison methods to regions associated with certain epigenetic markers, such as those where active or primed enhancers may be found.

At this stage, I am still developing my understanding of alignment free comparison methods that are available. I will attempt to identify cardiac enhancers using alignment free methods, incorporating methylation data.
If an extension to an existing method is developed or a completely new one is created, the aim is also to publish this methodology of process in a journal paper and any software to be available open source.
    
    \section{Timeline}
\definecolor{barblue}{RGB}{153,204,254}
\definecolor{groupblue}{RGB}{51,102,254}
\definecolor{linkred}{RGB}{165,0,33}


\begin{ganttchart}[
   canvas/.append style={fill=none, draw=black!5, line width=.75pt},
   hgrid style/.style={draw=black!5, line width=.75pt},
   vgrid,%={*1{draw=black!15, line width=.75pt}},
   today=0,
   today rule/.style={
     draw=black!64,
     dash pattern=on 3.5pt off 4.5pt,
     line width=1.5pt
   },
   today label font=,%\small\bfseries,
   title/.style={draw=none, fill=none},
   title label font=\bfseries,
   title label node/.append style={below=7pt},
   include title in canvas=false,
   % bar label font=\mdseries\small\color{black!70},
%     bar label node/.append style={align=left,text width=14em,below left=-5pt and 0pt},
   bar/.append style={fill=groupblue,draw=none},
   bar incomplete/.append style={fill=barblue},
   % bar progress label font=\mdseries\footnotesize\color{black!70},
   group incomplete/.append style={fill=groupblue},
   group left shift=0,
   group right shift=0,
   group height=.5,
   group peaks tip position=0,
   group label node/.append style={align=left,text width=15em},
        Mile2/.style={milestone/.append style={fill=yellow,shape=circle}}
   % group progress label font=\bfseries\small,
 ]{1}{12}
    \sffamily
 \gantttitle[
   title label node/.append style={below left=7pt and -3pt}
 ]{Months from now:\quad}{0}
 \gantttitlelist{0,2,...,24}{1} \\
    \ganttbar{Development of \texttt{R} library}{1}{3}
    \ganttmilestone[Mile2]{}{5}\\ 
%  \ganttbar{Model of nuclear calcium}{1}{2} \\
%     \ganttmilestone[Mile2]{}{9}
 \ganttbar{Alignment Based Enhancer Identification}{2}{5} 
 \ganttmilestone[Mile2]{}{6}
    \\
    % \ganttmilestone[Mile2]{}{10}
    % \ganttbar{Simplified model of channel interaction}{5}{7} \\
    % \ganttmilestone[Mile2]{}{12}
    \ganttbar{Alignment Free Enhancer Identification}{6}{9}
    \ganttmilestone[Mile2]{}{10} \\
    \ganttbar{Thesis write up}{10}{12}
    % \ganttlink{elem2}{elem3}

\end{ganttchart}\\
\caption{The period of work is indicated with blue bars while yellow circles indicate write up of the work for publication.}\\
    \bibliographystyle{acm}
\bibliography{LitReview}
\end{document}

Plan:
    Literature review
        Detail what enhancers are, what are their purpose 
        There's 