He2011coocupancy

A classical wet-lab technique for the identification of enhancers is the ChIP-seq of enhancer-associated proteins. 
This protocol is able to identify some, but not all of the enhancers that are active in tissue.

Old school standatrd ChIP-seq has been used to find active enhancers in the past, but the protocol has its weaknesses. A standard method is the identification of identifying enhancer related protein, but this method is only able to identify a subset of active enhancers in a a given tissue~\cite{heintzman2007distinct}.  

Researchers have modified existing protocols in attempts to overcome the downsides of protocols. This has been used to identify prospective cardiac regulatory sequences distinct from those found by standard protocols~\cite{he2011co}. What these guys also did is they looked at combinations of transcription factors to identify the enhancer regions active in the heart.

Heart stuff: In the field of cardiac stuff, people have had success with combining different types of data! They have combined ChIP-seq data for proteins associated with cartdiac TFs, computationally predicted enhancers, enhancer associated histone modification H3K4me1 (That's the same as ours!) as well as conservation to identify enhancer regions and link change of the regulatory landscape with congenital heart disease ~\cite{smemo2012regulatory}.  

