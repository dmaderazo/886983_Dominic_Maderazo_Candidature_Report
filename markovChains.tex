While Markov chains are not explicitly mentioned in most of the computational methods described above, a basic treatment of this mathematical object will be beneficial for the reader as some of the underlying techniques below rely on some of the properties of markov chains.

A Markov chain is a sequence $X_{t_1}, X_{t_2}, ...$ of random elements $X_i$ from some state space $\mathscr{X}$ with the following property
%such that the distribution of transition probabilities at the next state of the chain, conditional on all previous states, depends only on the current state of the chain; that is 
    \begin{equation}
        P(X_{t_{n+1}}|X_{t_n},X_{t_{n-1}},\ldots X_{t_1}) = P(X_{t_{n+1}}|X_{t_n});
    \end{equation}
this is known as the Markov property.
% {\color{red}
% {I'm not sure if the states should be upper case or lower case}}. 
%For illustrative purposes, here only discrete time Markov chains on discrete state spaces will be considered. 
% The motivating reason for the inclusion of Markov chains is because of Markov chain Monte Carlo (MCMC) being discussed below.
% The main idea behind MCMC is generating a Markov chain so that the limiting distribution of $X_t$ as $t\rightarrow\infty$ is some $\pi$. 
% In practice, this $\pi$ is the posterior distribution obtained for some quantities of interest in a Bayesian framework. 
% Ensuring that the distribution of $X$ converges to $\pi$, the chain must be constructed with certain properties. However, to describe these mathematically it is necessary to define the probability that the chain moves from state $i$ to state $j$ in $t$ steps as $ P_{ij}(t) = P(X_t = j | X_0 = i)$
% and the first return time to state $i$ to be $\tau_{ii} = \min\{t>0:X_t = i|X_0 = i\}$
%     \begin{enumerate}[(i)]
%         \item a \textbf{positive recurrent} Markov chain is one with the propery
%             \begin{enumerate}
%                 \item $E[\tau_{ii}] < \infty ,\, \forall i$
%             \end{enumerate}
        
%         \item A Markov chain is called \textbf{irreducible} if $\forall~i,j, \exists~t > 0$ such that $P_{ij}(t) > 0$.
%         % \item An irreducible chain is \textbf{recurrent} if $P(\tau_{ii} < \infty) = 1$ for some $i$.
%         % \item An irreducible recurrent chain is called \textbf{positive recurrent} if $E[\tau_{ii}]<\infty$ for some $i$.
%         \item An irreducible chain is \textbf{aperiodic} if for some $i$
%             $$gcd\{n>0:P(X_n = i|X_0 = i) > 0\} =1. $$
%     \end{enumerate}
% The irreducibility condition intuitively means that all states are reachable from any starting state in a finite number of steps, while requiring aperiodicity means that the chain will not oscillate between a subset of states in a periodic manner and positive recurrence guarantees that the chain will not drift to infinity. The combination of the three properties 